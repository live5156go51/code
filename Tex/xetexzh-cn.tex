\documentclass{article}
\usepackage{fontspec}
%%%%设置代码块
\usepackage{listings} 
\usepackage{xcolor}
\lstset{
numbers=left,
framexleftmargin=10mm,
frame=none,
backgroundcolor=\color[RGB]{245,245,244},
keywordstyle=\bf\color{blue},
identifierstyle=\bf,
numberstyle=\color[RGB]{0,192,192},
commentstyle=\it\color[RGB]{0,96,96},
stringstyle=\rmfamily\slshape\color[RGB]{128,0,0},
showstringspaces=false
}
%%%%%%%%%%%%%%%
\setmainfont{AR PL UKai CN}
\begin{document}
\title{TeXlive 2011安装使用}
\author{李俊鹏}
\maketitle{}
\newpage{}
 \section{简介}
 TEX Live 是一个完整的 TEX 系统,它的主页在 http://tug.org/texlive/
  \section{下载TeXlive2011}
  厦门大学开源软件镜像:http://mirrors.xmu.edu.cn/CTAN/systems/texlive/Images/
\section{安装}
  命令挂载到本地磁盘: sudo mount -t iso9660 -o loop /pathto/texlive2011-20110705.iso  /mnt/ \\
  图形介面安装:sudo ./install-tl -gui
\section{关于环境变量设置}
如果安装时候选择了在标准路径下创建符号连接,那就不需要设置环境变量了。否 \\
则,应该将使用的平台下TeX二进制程序的目录加入搜索路径,编辑用户目录 \~/.bash\_profile \\
PATH=/usr/local/texlive/2011/bin/i386-linux:\$PATH; export PATH \\
MANPATH=/usr/local/texlive/2011/texmf/doc/man:\$MANPATH; export MANPATH\\ 
INFOPATH=/usr/local/texlive/2011/texmf/doc/info:\$INFOPATH; export INFOPATH\\

\section{解决XeTeX中文配置}
参照TEX Live 指南 TEX Live 2011:\\
1. 将 texlive-fontconfig.conf 文件复制到 /etc/fonts/conf.d/09-texlive.conf。\\
2. 运行 fc-cache -fsv。\\
以下是测试XeTeX中文环境代码。
\begin{lstlisting}
\documentclass{article}
\usepackage{fontspec}
\setmainfont{WenQuanYi Zen Hei Sharp}
\begin{document}
 Hello,中文
\end{document}
\end{lstlisting}
第2行使用fontspec宏包以使用setmainfont命令设置导言区字体。\\
linux 系统用命令:fc-list :lang=zh-cn 查看setmainfont可设置的中文字体。\\
另外可以从Windows/Fonts/里复制些中文字体到/usr/share/fonts,\\
然后再用命令fc-cache更新字体系统,这样fc-list就可以识别了。
\section{XeTex beamer做slide}
\begin{lstlisting}
\documentclass{beamer}          %幻灯片宏包
\usepackage{beamerthemesplit}   %主题风格
\usepackage{fontspec}
\setmainfont{AR PL UKai CN}
\setsansfont{WenQuanYi Zen Hei Sharp}
\usepackage{graphics}
\begin{document}
  \title{创建slide}     %添加标题
  \subtitle{副标题}           %附标题
  \author{李俊鹏}           %添加作者
  \institute{地址、学校、专业}
  \date{\today{}}                 %添加日期
  \frame{\titlepage{}}            %添加主题幻灯片
  \frame{\frametitle{第一张幻灯片标题}
    \pause{}
    \begin{itemize}
    \item 第一名
    \end{itemize}
  }
\end{document}  
\end{lstlisting}
很多时候排Beamer幻灯片时,中文汉字显示为空白,是因为因为 beamer 默认使用 sans 字体。所以加上第五行setsansfont字体即可解决中文beamer。
\end{document}
