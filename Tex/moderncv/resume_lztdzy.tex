\documentclass[11pt,a4paper,xetex]{moderncv}
\usepackage{fontspec,xunicode, xltxtra}% 加载 xetex一些特殊字符和logo 的宏包
% character encoding
\defaultfontfeatures{Scale=MatchLowercase} % 这个参数保证 serif、sans-serif 和 monospace 字体在小写时大小匹配
\setmainfont[Numbers=OldStyle,Mapping=tex-text]{AR PL UMing CN}% 使用宋体,把数字设置为等宽,比较好看
\setsansfont[Mapping=tex-text]{AR PL UMing CN}
\setmonofont{AR PL UMing CN}
\usepackage[slantfont,boldfont,CJKnumber]{xeCJK}  % 加载 xeCJK,允许斜体、粗体和 CJK 数字以及 CJK 对空格的设置
% 设置中文字体
% ==========================================================
\setCJKmainfont[BoldFont=Adobe Heiti Std,ItalicFont=Adobe Kaiti Std]{Adobe Song Std}
\setCJKsansfont{Adobe Kaiti Std} %调整文章字体
\setCJKmonofont{Adobe Fangsong Std}
 
\setCJKfamilyfont{zhsong}{Adobe Song Std}
\setCJKfamilyfont{zhhei}{Adobe Heiti Std}
\setCJKfamilyfont{zhfs}{Adobe Fangsong Std}
\setCJKfamilyfont{zhkai}{Adobe Kaiti Std}
% Custom setting for 隶书 and 幼圆
\setCJKfamilyfont{zhli}{LiSu}
\setCJKfamilyfont{zhyou}{YouYuan}
\setCJKfamilyfont{zhfzyaoti}{FZYingBiXingShu-S16T}%FZYaoTi} %方正字体
\setCJKfamilyfont{hhahahhah}{STXingkai}

\newcommand*{\songti}{\CJKfamily{zhsong}} % 宋体
\newcommand*{\heiti}{\CJKfamily{zhhei}}   % 黑体
\newcommand*{\kaishu}{\CJKfamily{zhkai}}  % 楷书
\newcommand*{\fangsong}{\CJKfamily{zhfs}} % 仿宋
\newcommand*{\lishu}{\CJKfamily{zhli}}    % 隶书
\newcommand*{\youyuan}{\CJKfamily{zhyou}} % 幼圆
\newcommand*{\fzyaoti}{\CJKfamily{zhfzyaoti}}

%定义字号命令
\newcommand{\erhao}{\fontsize{22pt}{\baselineskip}\selectfont}
\newcommand{\xiaoerhao}{\fontsize{18pt}{\baselineskip}\selectfont}
\newcommand{\sanhao}{\fontsize{16pt}{\baselineskip}\selectfont}
\newcommand{\xiaosanhao}{\fontsize{15pt}{\baselineskip}\selectfont}
\newcommand{\sihao}{\fontsize{14pt}{\baselineskip}\selectfont}
\newcommand{\xiaosihao}{\fontsize{12pt}{\baselineskip}\selectfont}
\newcommand{\wuhao}{\fontsize{10.5pt}{\baselineskip}\selectfont}
\newcommand{\xiaowuhao}{\fontsize{9pt}{\baselineskip}\selectfont}
\newcommand{\liuhao}{\fontsize{7.5pt}{\baselineskip}\selectfont}
\newcommand{\temptest}{\fontsize{13pt}{\baselineskip}\selectfont} %测试符合小标题
% ==========================================================
\XeTeXlinebreaklocale "zh"
\XeTeXlinebreakskip=0pt plus 1pt minus 0.1pt
\widowpenalty=10000
\moderncvtheme[blue]{casual}  % optional argument are 'blue' (default), 'orange', 'red', 'green', 'grey' and 'roman' (for roman fonts, instead of sans serif fonts)
% adjust the page margins
%\moderncvtheme[green]{classic}                % idem      
\usepackage[scale=0.85]{geometry}
%\setlength{\hintscolumnwidth}{3cm}                                             % if you want to change the width of the column with the dates
%\AtBeginDocument{\setlength{\maketitlenamewidth}{6cm}}  % only for the classic theme, if you want to change the width of your name placeholder (to leave more space for your address details
\AtBeginDocument{\recomputelengths}                     % required when changes are made to page layout lengths
% personal data
% \firstname{John}
% \familyname{Doe}
% \title{Resumé title (optional)}               % optional, remove the line if not wanted
% \address{street and number}{postcode city}    % optional, remove the line if not wanted
% \mobile{mobile (optional)}                    % optional, remove the line if not wanted
% \phone{phone (optional)}                      % optional, remove the line if not wanted
% \fax{fax (optional)}                          % optional, remove the line if not wanted
% \email{email (optional)}                      % optional, remove the line if not wanted
% \homepage{homepage (optional)}                % optional, remove the line if not wanted
% \extrainfo{additional information (optional)} % optional, remove the line if not wanted
% \photo[64pt]{picture}                         % '64pt' is the height the picture must be resized to and 'picture' is the name of the picture file; optional, remove the line if not wanted
%\quote{Some quote (optional)}                 % optional, remove the line if not wanted
%\firstname{\fzyaoti {李俊鹏}}
%\familyname{柳州铁道职业技术学院
%\firstname{柳州铁道职业技术学院}
\firstname{毕业生就业推荐表}
%\brief{\small{学无止境!}}
\title{个人简历}
\address{柳州铁道职业技术学院}{545007}
\mobile{13978080642}
%\phone{13978080642}
\email{junstrix@gmail.com}
\homepage{www.imljp.info}
%\extrainfo{中青团员}
\photo[32pt]{lztdzy.jpeg}
%\quote{\temptest{\songti{毕业生就业推荐表}}}
\quote{\temptest{\songti{柳州铁道职业技术学院\qquad{通信技术专业\qquad{大专学历}}}}} %字体设置
%\nopagenumbers{}                             % uncomment to suppress automatic page numbering for CVs longer than one page
%----------------------------------------------------------------------------------
%            content
%----------------------------------------------------------------------------------
\begin{document}
\maketitle
\section{基本信息}
\cvcomputer{姓名:}{李俊鹏}{性别:}{男}
\cvcomputer{民族:}{汉族}{籍贯:}{广西玉林}
\cvcomputer{出生日期:}{1990-07-23}{政治面貌:}{共青团员}
\section{受教育经历}
%\cventry{2006--2009}{\textit{高中}}{博白县第三高级中学}{广西博白县}{}{}
\cvline{2006--2009}{\textit{高中},博白县第三高级中学,广西博白县}
\cvline{2009--\qquad}{\textit{大专},柳州铁道职业技术学院,广西柳州市}
\section{个人能力}
\cvline{学习能力}{在校学习期间考试成绩优秀,多次获得奖学金和参加各种技能比赛,有较强的自学能力;}
\cvline{理论水平}{在计算机领域、操作系统应用、计算机网络技术、自由软件,单片机及电子自动化,在通信领域,路由交换技术,通信线路,Voip,三网融合等相关领域,有一定的理论基础;}
\cvline{专业技能}{ 掌握C语言,VHDL等程序设计语言,能熟练使用OFFICE办公软件,能熟练操作linux操作系统;}
\cvline{工作能力}{工作尽职尽责、服从领导,能够认真、按时完成任务,具有良好的团队合作精神;}
\section{获奖经历}
\cvlistdoubleitem[\Neutral]{“华为杯”三网融合与网络优化二等奖}{柳州市网络操作比赛第二名}
\cvlistitem[\Neutral]{第四届“高教社杯”广西大学生电子设计大赛二等奖}
%\section{工作经历}
%\subsection{全职工作}
%\cvline{2004--2006}{资料员,中建七局一公司郑州分公司,河南省新乡市}%{Description line 1\newline{}Description line 2}% arguments 3 to 6 are optional
\section{爱好}
\cvlistdoubleitem[\Neutral]{篮球}{乒乓球}
\cvlistdoubleitem[\Neutral]{音乐}{Geek文化}
\section{自我评价}
\cvline{性格}{开朗、活泼、乐观}
\cvline{学习}{勤奋严谨,有钻研精神,动手能力强,有很强的自学能力,热衷于开源文化}
\cvline{生活}{勤俭节约,吃苦耐劳,热爱集体,乐于助人,积极参加各类公共活动,有良好的人际关系}
\cvline{工作}{做事细心,认真负责,能够高效率的完成工作,易于沟通,具有良好的创新意识和团队意识,热爱自己从事的工作,敢挑重担}
% Publications from a BibTeX file
%\newcommand{\refname}{发表论文}
%\nocite{*}
%\bibliographystyle{unsrt}
%\bibliography{hht}       % 'publications' is the name of a BibTeX file
%\begin {thebibliography}{99}
%  \bibitem {1} Hongtao Huang, Shaobin Huang, Tao Zhang. A Formal Method for Verifying Production Knowledge Base[A]. Internet Computing for Science and Engineering (ICICSE), 2009 Fourth International Conference on[C]. 2009: 19-23.
%  \bibitem {2} Tao Zhang, Shaobin Huang, Hongtao Huang. An Operational Semantics for UML RT-Statechart in Model Checking Context[A]. Internet Computing for Science and Engineering (ICICSE), 2009 Fourth International Conference on[C]. 2009: 12-18.
%  \bibitem {3} Mingyu Ji, Shaobin Huang, Hongtao Huang, et al. The New Development of PRISM for Probabilistic Model Checking[A]. Internet Computing for Science and Engineering (ICICSE), 2009 Fourth International Conference on[C]. 2009: 64-66.
%  \bibitem {4} 黄宏涛, 黄少滨, 张涛. 基于不动点计算的产生式知识库形式化建模方法[J]. 哈尔滨工程大学学报, .
% \end{thebibliography}
\end{document}
%% end of file `resume.tex'.

