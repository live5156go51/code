%% start of file `template-zh.tex'.
%% Copyright 2006-2012 Xavier Danaux (xdanaux@gmail.com).
%
% This work may be distributed and/or modified under the
% conditions of the LaTeX Project Public License version 1.3c,
% available at http://www.latex-project.org/lppl/.


\documentclass[11pt,a4paper,sans,xetex]{moderncv}   % possible options include font size ('10pt', '11pt' and '12pt'), paper size ('a4paper', 'letterpaper', 'a5paper', 'legalpaper', 'executivepaper' and 'landscape') and font family ('sans' and 'roman')

% moderncv 主题
\moderncvstyle{classic}                        % 选项参数是 ‘casual’, ‘classic’, ‘oldstyle’ 和 ’banking’
\moderncvcolor{blue}                          % 选项参数是 ‘blue’ (默认)、‘orange’、‘green’、‘red’、‘purple’ 和 ‘grey’
%\nopagenumbers{}                             % 消除注释以取消自动页码生成功能

% 字符编码
\usepackage{fontspec,xunicode,xltxtra}
\defaultfontfeatures{Scale=MatchLowercase} % 这个参数保证 serif、sans-serif 和 monospace 字体在小写时大小匹配
%\setmainfont[Mapping=tex-text]{AR PL UKai CN}
\setsansfont[Numbers=OldStyle,Mapping=tex-text]{AR PL UKai CN}% 使用宋体,把数字设置为等宽,比较好看
%\setmonofont{AR PL UMing CN}

\XeTeXlinebreaklocale "zh"
\XeTeXlinebreakskip=0pt plus 1pt minus 0.1pt
\widowpenalty=10000

% 调整页边距
\usepackage[scale=0.85]{geometry}
%\setlength{\hintscolumnwidth}{3cm}           % 如果你希望改变日期栏的宽度

% 个人信息
\firstname{\fontspec{FZYingBiXingShu-S16T}{李俊鹏}}
\familyname{}
\title{个人简历}                      % 可选项、如不需要可删除本行
\address{柳州铁道职业技术学院}{545007}             % 可选项、如不需要可删除本行
\mobile{13978080642}                         % 可选项、如不需要可删除本行
%\phone{+2~(345)~678~901}                          % 可选项、如不需要可删除本行
%\fax{+3~(456)~789~012}                            % 可选项、如不需要可删除本行
\email{junstrix@gmail.com}                    % 可选项、如不需要可删除本行
\homepage{junstrix.sinaapp.com}                  % 可选项、如不需要可删除本行
%\extrainfo{附加信息 (可选项)}                  % 可选项、如不需要可删除本行
\photo[50pt]{face}
%\photo[64pt][0.4pt]{picture}                  % ‘64pt’是图片必须压缩至的高度、‘0.4pt‘是图片边框的宽度 (如不需要可调节至0pt)、’picture‘ 是图片文件的名字;可选项、如不需要可删除本行
%\quote{引言(可选项)}                           % 可选项、如不需要可删除本行

% 显示索引号;仅用于在简历中使用了引言
%\makeatletter
%\renewcommand*{\bibliographyitemlabel}{\@biblabel{\arabic{enumiv}}}
%\makeatother


% 分类索引
%\usepackage{multibib}
%\newcites{book,misc}{{Books},{Others}}
%----------------------------------------------------------------------------------
%            内容
%----------------------------------------------------------------------------------
\begin{document}
\maketitle

\section{基本信息}
\cvcomputer{姓名:}{李俊鹏}{性别:}{男}
\cvcomputer{民族:}{汉族}{籍贯:}{广西玉林}
\cvcomputer{学历:}{大专}{专业:}{通信技术}
\cvcomputer{出生日期:}{1990-07-23}{政治面貌:}{共青团员}
\section{受教育经历}
\cvline{2006--2009}{\textit{高中},博白县第三高级中学,广西博白县}
\cvline{2009--\qquad}{\textit{大专},柳州铁道职业技术学院,广西柳州市}
\section{个人能力}
\cvline{学习能力}{在校学习期间考试成绩优秀,多次获得奖学金和参加各种技能比赛,有较强的自学能力;}
\cvline{理论水平}{在计算机领域、操作系统应用、计算机网络技术、自由软件,单片机及电子自动化,在通信领域,路由交换技术,通信线路,Voip,三网融合等相关领域,有一定的理论基础;}
\cvline{专业技能}{ 掌握C语言,VHDL等程序设计语言,能熟练使用Windows产品,能熟练操作Linux操作系统;}
\cvline{工作能力}{工作尽职尽责、服从领导,能够认真、按时完成任务,具有良好的团队合作精神;}
\section{获奖经历}
\cvlistitem{全国职业技能竞赛“华为杯”三网融合与网络优化二等奖}
\cvlistitem{第四届“高教社杯”广西大学生电子设计大专科组赛二等奖}
\cvlistitem{柳州市网络操作比赛第二名}
\cvlistitem{学院大学生电子设计竞赛一等奖}
\cvlistitem{学院网络操作比赛三等奖}
\section{爱好}
\cvlistdoubleitem{篮球}{乒乓球}
\cvlistdoubleitem{音乐}{开源文化}
\section{自我评价}
\cvline{性格}{开朗、活泼、乐观}
\cvline{学习}{勤奋严谨,有钻研精神,动手能力强,有很强的自学能力,热衷于开源文化}
\cvline{生活}{勤俭节约,吃苦耐劳,热爱集体,乐于助人,积极参加各类公共活动,有良好的人际关系}
\cvline{工作}{做事细心,认真负责,能够高效率的完成工作,易于沟通,具有良好的创新意识和团队意识,热爱自己从事的工作,敢挑重担}
\end{document}
%% 文件结尾 `template-zh.tex'.
