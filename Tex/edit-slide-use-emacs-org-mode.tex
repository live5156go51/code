% Created 2012-06-20 三 20:22
\documentclass[bigger]{beamer}
\usepackage[utf8]{inputenc}
\usepackage[T1]{fontenc}
\usepackage{fixltx2e}
\usepackage{graphicx}
\usepackage{longtable}
\usepackage{float}
\usepackage{wrapfig}
\usepackage{soul}
\usepackage{textcomp}
\usepackage{marvosym}
\usepackage{wasysym}
\usepackage{latexsym}
\usepackage{amssymb}
\usepackage{hyperref}
\usepackage{beamerthemesplit}   
\usepackage{fontspec}

\setsansfont{Adobe Kaiti Std}
\XeTeXlinebreaklocale "zh"
\XeTeXlinebreakskip=0pt plus 1pt minus 0.1pt
\tolerance=1000
\mode<beamer>{\usetheme{Madrid}}
\providecommand{\alert}[1]{\textbf{#1}}

\title{使用emacs编写slide in org-mode}
\author{Junstrix}
\date{2011-09-08 四}

\begin{document}

\maketitle

\begin{frame}
\frametitle{Outline}
\setcounter{tocdepth}{3}
\tableofcontents
\end{frame}


\section{安装更高的版本支持}
\label{sec-1}
\begin{frame}[fragile]
\frametitle{安装 org-mode}
\label{sec-1-1}

\begin{itemize}
\item 下载
\item 解压安装
     tar xvf org-xxx.tar -C \~/path/to/plugins

     make 

     make install
\item 设置.emacs
\end{itemize}

\begin{verbatim}
(setq load-path (cons "~/.emacs.d/plugins/org-7.7/lisp" load-path))
(setq load-path (cons "~/.emacs.d/plugins/org-7.7/contrib" load-path))
(require 'org-install)
(add-to-list 'auto-mode-alist '("\\.org\\'" . org-mode))
(global-set-key "\C-cl" 'org-store-link)
(global-set-key "\C-cc" 'org-capture)
(global-set-key "\C-ca" 'org-agenda)
(global-set-key "\C-cb" 'org-iswitchb)
\end{verbatim}
\end{frame}
\section{org-mode beamer相关配置}
\label{sec-2}
\begin{frame}[fragile]
\frametitle{org-mode beamer}
\label{sec-2-1}

一个slide首先设置slide相关属性,包括标题、作者、日期、Email等
\begin{itemize}
\item C-c C-e t 插入并编辑标题、作者、日期等
\item beamer环境配置(org-insert-beamer-options-template)
\end{itemize}

\begin{verbatim}
#+startup: beamer
#+LaTeX_CLASS: beamer
#+LaTeX_CLASS_OPTIONS: [bigger]
#+LATEX_FRAME_LEVEL: 2
#+COLUMNS: %40ITEM %10BEAMER_env(Env) %9BEAMER_envargs(Env Args)
%4BEAMER_col(Col) %10BEAMER_extra(Extra)
\end{verbatim}
\end{frame}
\section{一个slide例子}
\label{sec-3}
\begin{frame}
\frametitle{A Simple Slide Example}
\label{sec-3-1}

This slide consists of some text with a number of bullet points:

\begin{itemize}
\item the first, very @important@, point!
\item the previous point shows the use of the special markup which
  translates to the Beamer specific \emph{alert} command for highlighting
  text.
\end{itemize}


The above list could be numbered or any other type of list and may
include sub-lists.
\end{frame}
\section{如何使用xelatex编译org-mode导出的Tex源码}
\label{sec-4}
\begin{frame}[fragile]
\frametitle{xelatex编译\TeX{}}
\label{sec-4-1}

因为emacs导出没有直接xelatex的支持,而XeTeX的优点就不说了,所以我们要自己修改
emacs in org-mode 导出的\TeX{}源码,以Xelatex能正确编译
\begin{itemize}
\item 增加必要的宏包
\end{itemize}

\begin{verbatim}
\usepackage{fontspec}
\usepackage{hyperref}  %bookmark and add it!!!
\setsansfont{Adobe Kaiti Std} %command list fonts:fc-list :lang=zh-cn
\XeTeXlinebreaklocale "zh"
\XeTeXlinebreakskip=0pt plus 1pt minus 0.1pt
\widowpenalty=10000
\end{verbatim}


xetex导盲区设置请参照链接:\href{https://github.com/live5156go51/code/blob/master/Tex/moderncv/resume.tex}{xetex配置例子}
\begin{itemize}
\item 编译 xelatex file.tex
\item 官方教程 \href{http://orgmode.org/worg/org-tutorials/org-beamer/tutorial.html}{http://orgmode.org/worg/org-tutorials/org-beamer/tutorial.html}
\end{itemize}
\end{frame}
\section{更漂亮的slide multiplexer}
\label{sec-5}
\begin{frame}
\frametitle{A more complex slide}
\label{sec-5-1}

This slide illustrates the use of Beamer blocks.  The following text,
with its own headline, is displayed in a block:

使用属性快捷键:C-c C-b(org-beamer-select-environment)
\begin{theorem}[Org mode increases productivity]
\label{sec-5-1-1}

\begin{itemize}
\item org mode means not having to remember \LaTeX{} commands.
\item it is based on ascii text which is inherently portable.
\item Emacs!
\end{itemize}

    \hfill \(\qed\)
\end{theorem}
\end{frame}
\begin{frame}
\frametitle{两列}
\label{sec-5-2}
\begin{columns}
\begin{column}{0.4\textwidth}
%% 文字块
\label{sec-5-2-1}

\begin{itemize}
\item 这个slide由两列组成
\item 第一列没有heading和文字
\item 第二列包含一幅图片和例子文字块
\end{itemize}
\end{column}
\begin{column}{0.6\textwidth}
\begin{example}[一幅图片]
\label{sec-5-2-2}

 \includegraphics[width=.9\linewidth]{//tmp/backgroup01.jpg}
\end{example}
\end{column}
\end{columns}
\end{frame}
\begin{frame}[fragile,t]
\frametitle{代码演示}
\label{sec-5-3}
\begin{columns}
\begin{column}{0.45\textwidth}
\begin{block}{Octave 源代码}
\label{sec-5-3-1}


\begin{verbatim}
A = [1 2 ; 3 4]
b = [1; 1];
x = A\b
\end{verbatim}
\end{block}
\end{column}
\begin{column}{0.4\textwidth}
\begin{block}<2->{输出}
\label{sec-5-3-2}



\begin{verbatim}
A =

   1   2
   3   4

x =

  -1
   1
\end{verbatim}
\end{block}
\end{column}
\end{columns}
\end{frame}
\section{This is the first structural section}
\label{sec-6}

     
\begin{frame}
\frametitle{Frame 1}
\framesubtitle{with a subtitle}
\label{sec-6-1}
\begin{columns}[t]
\begin{column}{0.5\textwidth}
\begin{block}{Thanks to Eric Fraga}
\label{sec-6-1-1}

    for the first viable beamer setup in Org
\end{block}
\end{column}
\begin{column}{0.5\textwidth}
\begin{block}<2->{Thanks to everyone else}
\label{sec-6-1-2}

    for contributing to the discussion
\note{This will be formatted as a beamer note
}
\end{block}
\end{column}
\end{columns}
\end{frame}
\begin{frame}
\frametitle{Frame 2}
\framesubtitle{where we will not use columns}
\label{sec-6-2}
\begin{block}{Request}
\label{sec-6-2-1}

    Please test this stuff!
\end{block}
\end{frame}

\end{document}