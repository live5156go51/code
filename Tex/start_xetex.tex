\documentclass{beamer}          
\usepackage{beamerthemesplit}   
\usepackage{fontspec}
\usepackage{graphics}
\setsansfont{Adobe Kaiti Std}
\XeTeXlinebreaklocale "zh"
\XeTeXlinebreakskip=0pt plus 1pt minus 0.1pt
\begin{document}
  \title{从安装TeXlive到使用}     
  \subtitle{开源盛世,人心所向}   
  \author{Junstrix}           
  \institute{Email:junstrix@gmail.com}
  \date{\today{}}             
  \frame{\titlepage{}}        
  \frame{\frametitle{大致安装使用步骤}
    \pause{}
    \begin{itemize}
    \item 下载
    \item 安装
    \item XeTex中文配置
    \item 其它扩展字体下载与使用
    \item 一些示例
    \end{itemize}
  }
  \frame{\frametitle{下载}
    \pause{}
    \begin{itemize}
      \item 官网:http://www.tug.org/texlive
      \item 下载网络安装包:\\
        http://mirrors.xmu.edu.cn/CTAN/systems/texlive/tlnet/
      \item 下载光盘镜像包: \\
        http://mirrors.xmu.edu.cn/CTAN/systems/texlive/Images/
    \end{itemize} 
  }
  \frame{\frametitle{安装}
    \pause{} 
    \begin{itemize}
      \item 安装方式:1.网络安装 2.本地安装      
        选择本地安装只需要下载解压 install-tl-unx.tar.gz\\
        执行./install-tl -gui 根据选项图形安装 \\
        如果本地安装方式,首先下载光盘镜像文件或光盘刻录\\
        mount挂载到本地,同样执行./install-tl -gui安装
      \item 注意问题 \\
        1.环境变量的设置 \\
        2.与旧版本冲突问题
    \end{itemize}
  }
  \frame{\frametitle{XeTex中文配置}
    \pause{}
    \begin{itemize}
    \item 
      默认情况下XeTeX随TeXlive发行版一同安装,只需要配置下就可以直接应用系统字体来排版中文。
    \item 解决XeTex中文问题 \\
      1. 将 texlive-fontconfig.conf 文件复制到 /etc/fonts/conf.d/09-texlive.conf。\\
      2. 运行 fc-cache -fsv。\\
    \item 相关命令 \\
      1.fc-list 查看系统已经安装字体。fc-list :lang=zh-cn可以查看系统已安装的中文字体 \\
      2.fc-cache 更新字体。fc-list -fsv,会扫描字体目录从而全面更新系统字体。
    \end{itemize}
  }
  \frame{\frametitle{中文测试}
    \pause{}
    \begin{itemize}
    \item 测试代码 \\
      $\backslash$ documentclass\{article\} \\
      $\backslash$ usepackage\{fontspec\} \\
      $\backslash$ setmainfont\{AR PL UKai CN\} \\ 
      \% AR PL UKai CN即用fc-list :lang=zh-cn命令列出来的系统中文字体 \\
      $\backslash$ begin\{document\}\\
        \qquad{}中文测试\\
      $\backslash$ end\{document\}
    \end{itemize}
  }
  \frame{\frametitle{其它字体下载与使用}
    \pause{}
    \begin{itemize}
    \item 字体下载 \\
      一个方法是直接复制Windows格式为ttf的字体 /Windows/Fonts \\
      如:cp pathto/Windows/Fonts/*.ttf  /usr/share/fonts/win-ttf(自行创建) \\
      另一个方法就是网络资源,一些私权字体比如方正字体、adobe字体
    \item 使用 \\
      首先fc-cache -fsv更新系统字体,然后fc-list :lang=zh-cn查看是否字体更新成功。\\
      系统字体更新正常XeTeX即可直接调用了。
    \end{itemize}
  }
  \frame{\frametitle{示例}
    \begin{itemize}
    \item XeTex beamer slide 
    \item XeTex moderncv个人简历
    \item XeTex高层中文解决方案
    \item XeTex低层中文解决方案
    \end{itemize}    
  }
\end{document}
