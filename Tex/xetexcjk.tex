% 高层中文解决方案的 TeX 源码
%中文字体复制到/usr/share/fonts/win-ttf
%命令:fc-cache -fsv
%需要root权限才能正常编译
\documentclass[UTF8,adobefonts]{ctexart}
\usepackage{fontspec}
\usepackage{xunicode}
\usepackage{xltxtra}
\title{Texlive 又一中文解决方案}
\author{Junstrix}
\date{\today}

\begin{document}
  \maketitle{}


%  \begin {flushleft}
%     1. 安装 LaTeX

% 推荐从 iso 镜像安装(而非用系统的包管理工具从源里安装) TeX Live 2009 ,具体请参考我的这篇 wiki 文章。

% 2. 安装必需的字体

% 推荐使用 Adobe 的四款中文字体(宋体、黑体、仿宋、楷体)加 Windows 下的两款字体(隶书、幼圆)。我把它们收集好并打包了,您可以从这儿下载。解压之后,把字体文件扔到 ~/.fonts 或者/usr/share/fonts 下面就行了。

% 执行 \$ fc-list | grep -i adobe 来确认安装成功。

% 3. 修改配置文件

% 为了让后面的高层方案支持隶书和幼圆字体,修改 ctex 宏包里的这个配置文件 /usr/local/texlive/2009/texmf-dist/tex/latex/ctex/fontset/ctex-xecjk-adobefonts.def ,在 “\backslash setCJKfamilyfont\{zhkai\}\{Adobe Kaiti Std\}”  \\
%  这一行后面添加以下内容:

% \backslash setCJKfamilyfont\{zhli\}\{LiSu\} \\
% \backslash setCJKfamilyfont\{zhyou\}\{YouYuan\}
% 4. 开始写文档吧~

% “基于 XeTeX 的方案”中,又可以分为高层和低层两种方案。如果您想写中文文档,那么用高层方案会比较合适;如果您只是想在英文文档中穿插中文,那么低层方案更加合适。

% 对这两种方案,我分别写了简单的模板,用来演示中文的基本用法(如改变字体和字号)。注意,必须用 xelatex 这个命令编译。
%   \end{flushleft}

\begin{center}
  1. 字体示例:\\
  \begin{tabular}{c|c}
    \hline
    \textbf{\TeX 命令} & \textbf{效果}\\
    \hline
    \verb|{\songti 宋体}| & {\songti 宋体}\\
    \hline
    \verb|{\heiti 黑体}| & {\heiti 黑体}\\
    \hline
    \verb|{\fangsong 仿宋}| & {\fangsong 仿宋}\\
    \hline
    \verb|{\kaishu 楷书}| & {\kaishu 楷书}\\
    \hline
    \verb|{\lishu 隶书}| & {\lishu 隶书}\\
    \hline
    \verb|{\youyuan 幼圆}| & {\youyuan 幼圆}\\
    \hline
  \end{tabular}
\end{center}
 
\begin{center}
  2. 字号示例:\\
  {\zihao{0}初号}
  {\zihao{1}一号}
  {\zihao{2}二号}
  {\zihao{3}三号}
  {\zihao{4}四号}
  {\zihao{5}五号}
  {\zihao{6}六号}
  {\zihao{7}七号}
  {\zihao{8}八号}
\end{center}

\end{document}
